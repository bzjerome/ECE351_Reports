%%%%%%%%%%%%%%%%%%%%%%%%%%%%%%%%%%%%%%%%%%%%%
%					    %
%	Brady Jerome			    %
%	ECE 351-52			    %
%	Lab 2				    %
%	2/13/2020			    %
%	Introduction to convolution coding  %
%					    %
%%%%%%%%%%%%%%%%%%%%%%%%%%%%%%%%%%%%%%%%%%%%%

\documentclass[12pt]{report}
\usepackage[english]{babel}
%\usepackage{natbib}
\usepackage{url}
\usepackage[utf8x]{inputenc}
\usepackage{amsmath}
\usepackage{graphicx}
\graphicspath{{images/}}
\usepackage{parskip}
\usepackage{fancyhdr}
\usepackage{vmargin}
\usepackage{listings}
\usepackage{hyperref}
\usepackage{xcolor}

\definecolor{codegreen}{rgb}{0,0.6,0}
\definecolor{codegray}{rgb}{0.5,0.5,0.5}
\definecolor{codeblue}{rgb}{0,0,0.95}
\definecolor{backcolour}{rgb}{0.95,0.95,0.92}

\lstdefinestyle{mystyle}{
    backgroundcolor=\color{backcolour},   
    commentstyle=\color{codegreen},
    keywordstyle=\color{codeblue},
    numberstyle=\tiny\color{codegray},
    stringstyle=\color{codegreen},
    basicstyle=\ttfamily\footnotesize,
    breakatwhitespace=false,         
    breaklines=true,                 
    captionpos=b,                    
    keepspaces=true,                 
    numbers=left,                    
    numbersep=5pt,                  
    showspaces=false,                
    showstringspaces=false,
    showtabs=false,                  
    tabsize=2
}
 
\lstset{style=mystyle}

\setmarginsrb{3 cm}{2.5 cm}{3 cm}{2.5 cm}{1 cm}{1.5 cm}{1 cm}{1.5 cm}

\title{Lab3 Report}								
% Title
\author{Brady Jerome}						
% Author
\date{2/10/2020}
% Date

\makeatletter
\let\thetitle\@title
\let\theauthor\@author
\let\thedate\@date
\makeatother

\pagestyle{fancy}
\fancyhf{}
\rhead{\theauthor}
\lhead{\thetitle}
\cfoot{\thepage}
%%%%%%%%%%%%%%%%%%%%%%%%%%%%%%%%%%%%%%%%%%%%
\begin{document}

%%%%%%%%%%%%%%%%%%%%%%%%%%%%%%%%%%%%%%%%%%%%%%%%%%%%%%%%%%%%%%%%%%%%%%%%%%%%%%%%%%%%%%%%%

\begin{titlepage}
	\centering
    \vspace*{0.5 cm}
   % \includegraphics[scale = 0.075]{bsulogo.png}\\[1.0 cm]	% University Logo
\begin{center}    \textsc{\Large   ECE 351 - Section 51 }\\[2.0 cm]	\end{center}% University Name
	\textsc{\Large Lab Title  }\\[0.5 cm]				% Course Code
	\rule{\linewidth}{0.2 mm} \\[0.4 cm]
	{ \huge \bfseries \thetitle}\\
	\rule{\linewidth}{0.2 mm} \\[1.5 cm]
	
	\begin{minipage}{0.4\textwidth}
		\begin{flushleft} \large
			\end{flushleft}
			\end{minipage}~
			\begin{minipage}{0.4\textwidth}
            
			\begin{flushright} \large
			\emph{Submitted By :} \\
			Brady Jerome  
		\end{flushright}
           
	\end{minipage}\\[2 cm]
	
    
    
    
    
	
\end{titlepage}

%%%%%%%%%%%%%%%%%%%%%%%%%%%%%%%%%%%%%%%%%%%%%%%%%%%%%%%%%%%%%%%%%%%%%%%%%%%%%%%%%%%%%%%%%

\tableofcontents
\pagebreak

%%%%%%%%%%%%%%%%%%%%%%%%%%%%%%%%%%%%%%%%%%%%%%%%%%%%%%%%%%%%%%%%%%%%%%%%%%%%%%%%%%%%%%%%%
\renewcommand{\thesection}{\arabic{section}}
\section{Introduction}

The goal of this lab is to introduce convolution through a set of functions using python. By making a step and ramp function, the convolution and derivative could also be found. Additionally, a sine and cosine function were found, as well as plots for the convolution with time shifts, time scaling, and time reversal.

\section{Equations}

\begin{equation}
	$ramp(t) - ramp(t-3) + 5*step(t-3) - 2*step(t-6) - 2*ramp(t-6)$
\end{equation}
	
\section{Methodology}

For this lab, most of the general code was given. This lab used the plotting feature the most. By defining the size, subplot, and axis, a plot of the desired result can be generated using plt.plot(). There is also the use of assigning functions and calling them up later. The most used case is to plot the function, but when using the ramp and step function to create a plot, they were called up later inside of the new function. To accomplish the tasks given, simply change the value of 't' in the plt.plot() function. This allows scaling, shifting, and reversal of the t-axis.


\section{Results}

The outcome of Lab 2 were as expected. The graphs are the main result, so by observing the graphs, any error can be found. The step and ramp functions were simple, while the plotting of the function and the derivative were a bit trickier. Overall, this lab went very smoothly and gave expected results.

\section{Error Analysis}

The derivative of the plotted function was difficult to create a working model for. The dy[:,0] term doesn't work properly and just causes a straight line to form on the graph. Taking it out resulted in a very long straight line. Neither of these results is the correct response for the derivative. 

\section{Questions}

1. Are the plots from Part 3 Task 4 and Part 3 Task 5 identical? Is it possible for them to match? Explain why or why not.
	A: The plots are not identical. While they are similar, they do not fully line up. This is most likely due to the software of np.diff() that calculates the values. This value is difficult to change as it is part of the softwares code and operates in its own way. The function can be altered a bit to account for this, which will get the plots closer together.
	
2. How does the correlation between the two plots (from Part 3 Task 4 and Part 3 Task 5) change if you were to change the step size within the time variable in Task 5? Explain why this happens.
	A: Changing the step size increases the resolution. By giving the program more increments, the graph will become smoother. This gives a greater resolution to the graph. Since the program uses increments to graph a line, giving it more increments causes the lines to get smaller and smaller, making the overall line smoother.
	
3. Leave any feedback on the clarity of lab tasks, expectations, and deliverables.
	A: Personally, this lab went very well. The instructions are easy to comprehend and the instructor does a very good job of explaining the lab instructions.

\section{Conclusion}

This lab showed a very useful insight into the plt.plot() function. Using this allowed the graphs to be created seamlessly. This lab gave a useful library of the step and ramp functions that can be used for later problems. Having the availability to check results using python is a very useful skill to have. With the introductory skills gained after lab 2, future labs and problems will be easier to understand and complete.

\end{document}
