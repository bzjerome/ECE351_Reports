%%%%%%%%%%%%%%%%%%%%%%%%%%%%%%%%%%%%%%%%%%%%%
%					    					%
%	Brady Jerome			    			%
%	ECE 351-52			    				%
%	Lab 6				    				%
%	3/12/2020			    				%
%	Partial Fraction Expansion			    %
%					   					    %
%%%%%%%%%%%%%%%%%%%%%%%%%%%%%%%%%%%%%%%%%%%%%

\documentclass[12pt]{report}
\usepackage[english]{babel}
%\usepackage{natbib}
\usepackage{url}
\usepackage[utf8x]{inputenc}
\usepackage{amsmath}
\usepackage{graphicx}
\graphicspath{{images/}}
\usepackage{parskip}
\usepackage{fancyhdr}
\usepackage{vmargin}
\usepackage{listings}
\usepackage{hyperref}
\usepackage{xcolor}

\definecolor{codegreen}{rgb}{0,0.6,0}
\definecolor{codegray}{rgb}{0.5,0.5,0.5}
\definecolor{codeblue}{rgb}{0,0,0.95}
\definecolor{backcolour}{rgb}{0.95,0.95,0.92}

\lstdefinestyle{mystyle}{
    backgroundcolor=\color{backcolour},   
    commentstyle=\color{codegreen},
    keywordstyle=\color{codeblue},
    numberstyle=\tiny\color{codegray},
    stringstyle=\color{codegreen},
    basicstyle=\ttfamily\footnotesize,
    breakatwhitespace=false,         
    breaklines=true,                 
    captionpos=b,                    
    keepspaces=true,                 
    numbers=left,                    
    numbersep=5pt,                  
    showspaces=false,                
    showstringspaces=false,
    showtabs=false,                  
    tabsize=2
}
 
\lstset{style=mystyle}

\setmarginsrb{3 cm}{2.5 cm}{3 cm}{2.5 cm}{1 cm}{1.5 cm}{1 cm}{1.5 cm}

\title{Lab6 Report}								
% Title
\author{Brady Jerome}						
% Author
\date{3/12/2020}
% Date

\makeatletter
\let\thetitle\@title
\let\theauthor\@author
\let\thedate\@date
\makeatother

\pagestyle{fancy}
\fancyhf{}
\rhead{\theauthor}
\lhead{\thetitle}
\cfoot{\thepage}
%%%%%%%%%%%%%%%%%%%%%%%%%%%%%%%%%%%%%%%%%%%%
\begin{document}

%%%%%%%%%%%%%%%%%%%%%%%%%%%%%%%%%%%%%%%%%%%%%%%%%%%%%%%%%%%%%%%%%%%%%%%%%%%%%%%%%%%%%%%%%

\begin{titlepage}
	\centering
    \vspace*{0.5 cm}
   % \includegraphics[scale = 0.075]{bsulogo.png}\\[1.0 cm]	% University Logo
\begin{center}    \textsc{\Large   ECE 351 - Section 51 }\\[2.0 cm]	\end{center}% University Name
	\textsc{\Large Lab Title  }\\[0.5 cm]				% Course Code
	\rule{\linewidth}{0.2 mm} \\[0.4 cm]
	{ \huge \bfseries \thetitle}\\
	\rule{\linewidth}{0.2 mm} \\[1.5 cm]
	
	\begin{minipage}{0.4\textwidth}
		\begin{flushleft} \large
			\end{flushleft}
			\end{minipage}~
			\begin{minipage}{0.4\textwidth}
            
			\begin{flushright} \large
			\emph{Submitted By :} \\
			Brady Jerome  
		\end{flushright}
           
	\end{minipage}\\[2 cm]
	
    
    
    
    
	
\end{titlepage}

%%%%%%%%%%%%%%%%%%%%%%%%%%%%%%%%%%%%%%%%%%%%%%%%%%%%%%%%%%%%%%%%%%%%%%%%%%%%%%%%%%%%%%%%%

\tableofcontents
\pagebreak

%%%%%%%%%%%%%%%%%%%%%%%%%%%%%%%%%%%%%%%%%%%%%%%%%%%%%%%%%%%%%%%%%%%%%%%%%%%%%%%%%%%%%%%%%
\renewcommand{\thesection}{\arabic{section}}
\section{Introduction}

This lab is used to understand Partial Faction Expansion using Python functions and testing them with hand-derived calculations and the cosine method. This lab uses functions such as: sig.step() and sig.residue(). Sig.residue() is used to calculate the (R, P, _) values for the lab, which are used in the cosine method to replace alpha, omega, magnitude, and angle. The (R, P, _) values are given in the results section.

\section{Equations}

\begin{equation1}
	$y''(t)+10*y'(t)+24*y(t)=x''(t)+6*x'(t)+12*x(t)$
\end{equation1}
	
Equation 1 is the differential equation from the prelab used to derive the transfer function and the time-domain response using partial fraction expansion.

\begin{equation2}
	$H(s)=((s^2)+6*s+12)/((s+6)*(s+4))$
	$f(t)=((1/2)+exp(-6*t)-(1/2)*exp(-4*t))$
\end{equation2}

The first equation is the transfer function, derived from the prelab. The second equation is the time-domain output, derived from the transfer function.

\begin{equation3}
	$y^5(t)+18*y^4(t)+218*y^3(t)+2036*y^2(t)+9085*y^1(t)+25250*y(t)=25250*x(t)$
\end{equation3}

This is a 5th order polynomial used in Part 2 of this lab. It is used to determine the (R, P, _) values, and to observe the sig.residue() function.
	
\section{Methodology}

Before this lab begins, find the transfer function for the differential equation given in the prelab. Then, use this equation to graph the step response using sig.step() and  using the time-domain transfer function by setting an array for the numerator and the denominator. Compare these graphs to each other (graphs should be identical). Then, find the transfer function step response for Part 2 of this lab. This is a 5th order differential equation, but it follows a similar procedure as before. First, find the (R, P, _) values for the equation. This is done using sig.residue(). Then, graph step response of the equation using the cosine method and sig.step(). Comparing these graphs should result in two identical graphs.

\section{Results}

\begin{figure1}
	\centering
	\includegraphics[width=0.7\linewidth]{C:/Part1Task1}
	\caption{}
	\label{fig:part1task1}
\end{figure1}

Figure 1 is the step response derived from the prelab equation

\begin{figure2}
	\centering
	\includegraphics[width=0.7\linewidth]{C:/Part1Task2}
	\caption{}
	\label{fig:part1task2}
\end{figure2}

Figure 2 above is the step response that used sig.step() to graph. This figure should be identical to figure 1

\begin{figure3}
	\centering
	\includegraphics[width=0.7\linewidth]{C:/Part2Task2}
	\caption{}
	\label{fig:part2task2}
\end{figure3}

Figure 3 is the time-domain response of the 5th order differential equation using the cosine method

\begin{figure4}
	\centering
	\includegraphics[width=0.7\linewidth]{C:/Part2Task3}
	\caption{}
	\label{fig:part2task3}
\end{figure4}

Figure 4 is the time-domain response using sig.step(). It should be identical to figure 3.

(R, P, _) values: 
R1: [1.  -0.5  0.5]
P1: [-6. -4.  0.]
R2: [ 1.        +0.j         -0.48557692+0.72836538j -0.48557692-0.72836538j
-0.21461963+0.j          0.09288674-0.04765193j  0.09288674+0.04765193j]
P2: [  0. +0.j  -3. +4.j  -3. -4.j -10. +0.j  -1.+10.j  -1.-10.j]

These are the (R,P,_) values that were found using sig.residue(). They are used for figures 3 and 4.

\section{Error Analysis}

 The error for this lab would come from entering the numerator or denominator array incorrectly into Python. For the residue, there needs to be an added “0” at the end of the denominator. This allows for the proper (R, P, _) values to be determined. Otherwise, the rest of the error would just be syntax or step errors. The final error would be my prelab. I did not derive the correct equation from the prelab, so I needed to use the TA’s equation that he gave to us. In the end, this is the only error present in my lab.

\section{Questions}

1. For a non-complex pole-residue term, you can still use the cosine method. Explain why this works.
	A1:Since these values have poles (the denominator such as (s+1)), then the cosine method is feasible. These poles determine where a value can be 0, so using the cosine method allows for a simple way to find the time-domain response of the differential equations.
	
2. Leave any feedback on the clarity of lab tasks, expectations, and deliverables.
	A2: I have no feedback. This lab was overall easy to understand and fun to work through.

\section{Conclusion}

This lab went very smoothly. The only problem I ran into was the sig.residue() values for the denominator. I forgot to include the “0” at the end, but I got it fixed with some help from the TA. Otherwise, this lab went very smoothly. The graphs are all correct (given by identical graphs) and I was able to obtain the values for (R, P, _) (which are given in the results section).

\end{document}